\section{Semántica}
\subsubsection*{Dirección por sintáxis}
      \PN Un conjunto de ecuaciones es \textit{dirigido por sintáxis} cuando se satisfacen las siguientes condiciones:
      \begin{itemize}
        \item hay una ecuación por cada producción de la gramática abstracta
        \item cada ecuación que expresa el significado de una frase compuesta, lo hace puramente en función de los significados de sus subfrases inmediatas
      \end{itemize}
    
    \subsubsection*{Composicionalidad}
      \PN Una semántica se dice que es \textit{composicional} cuando el significado de una frase no depende de ninguna propiedad de sus subfrases, salvo de sus significados.

    \vspace{5mm}
    \PN Dirección por sintáxis $\Rightarrow$ composicionalidad
    
    \subsubsection*{Sustitución}
      \begin{eqnarray*}
        \Delta &=& \lang{var} \rightarrow \lang{intexp} \\
        &\in& \lang{intexp} \times \Delta \rightarrow \lang{intexp} \\
        (Qv.b)/\delta &=& Qv_{new}.(b/[\delta|v:v_{new}]) \\ \\
        \text{donde} \; v_{new} &\notin& \bigcup_{\omega \in FV(b-\{v\})} FV(\delta \omega)
      \end{eqnarray*}
    
    \subsubsection*{Propiedades}
      \begin{itemize}
        \item \textbf{Teorema de Coincidencia:} expresa que el significado de una frase no puede depender de variables que no ocurran libres en la misma.
          \PN \textbf{Enunciado:} Si dos estados $\sigma, \sigma'$ coinciden en las variables libres de $p$, entonces da lo mismo evaluar $p$ en $\sigma$ o $\sigma'$.
          \[
            (\forall \omega \in FV(p) . \sigma \omega = \sigma' \omega) \Rightarrow \semsAGU{p} = \semsp{p}
          \]
        \item \textbf{Teorema de Renombre:} asegura que el significado no depende de las variables ligadas de una frase.
          \PN \textbf{Enunciado:} Los nombres de las variables ligadas no tienen importancia.
          \[
            u \notin FV(q) - \{v\} \Rightarrow \semAGU{\forall u . q/v \rightarrow u} = \semAGU{\forall v . q}
          \]
        \item \textbf{Teorema de Sustitución:} Si aplico la sustitución $\delta$ a $p$ y luego evaluo en el estado $\sigma$, puedo obtener el mismo resultado a partir de $p$ sin sustituir si evaluo en un estado que hace el trabajo de $\delta$ y de $\sigma$ (en las variables libres de $p$).
          \[
            (\forall \omega \in FV(p) . \semsAGU{\delta \; \omega} = \sigma' \omega) \Rightarrow \semsAGU{p/\delta} = \semsp{p}
          \]
      \end{itemize}