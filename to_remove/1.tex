\begin{section}
El CL puro sólo contiene variables, aplicaciones y la notación lambda (llamada abstracción).
\[
\begin{array}{llll}
\left\langle exp\right\rangle & ::= &  & \text{expresiones o términos}\\
 & & \  \var & \text{variables} \\
 & & \mid \ex \ex & \text{aplicación} \\
 & & \mid \lambda \var.\ \ex & \text{abstracción o expresión lambda} \\
 \end{array}
\]
\textbf{Convención:} La aplicación asocia a izquierda.
En $\lambda v.e$, la primer ocurrencia de $v$ es ligadora
y su alcance es $e$.
\smallskip
Por ejemplo,
\smallskip
\qquad\qquad\qquad$\lambda x.(\lambda y.xyx)x$
\smallskip
es lo mismo que
\smallskip
\qquad\qquad\qquad$\lambda x.(\lambda y.(xy)x)x$

\smallskip

\textbf{Variables libres:}
\[
\begin{array}{llllll}
FV(v)& = & \{v\} \smallskip\\
FV(ee')& =  &FV(e) \cup FV(e')\smallskip\\
FV(\lambda v.e)& = & FV(e)\ -\  \{v\}\ \smallskip\\
\end{array}
\]

\textbf{Conjunto de sustituciones:} $\Delta = \var \rightarrow \ex$

\smallskip

\textbf{Operador  sustitución:} $\_/\_ \in \ex \times\Delta \rightarrow \ex$
\[
\begin{array}{rll}
\uncover<3->{
v/\delta &  = &  \delta v \smallskip\\
  (ee')/\delta & =& (e/\delta)(e'/\delta)\smallskip\\}
  \uncover<4->{
(\lambda v. e)/\delta&  =&  \lambda v_{new}.\  e/[\delta|v: v_{new}]\\
&&\text{ donde
$v_{new} \not\in \bigcup_{w \in FV(e) -\{v\}} FV(\delta\ w)$}}
\end{array}
\]

\smallskip

\textbf{Renombre: } Cambio en $\lambda v.e$ de la
  variable ligada $v$ (y todas sus ocurrencias) por una variable $v'$
  que no ocurra libre en $e$: $\lambda v'.\ e/v\mapsto v'$
 donde $v'\notin FV(e)$.

\smallskip

$\alpha$-\textbf{conversión}: Si $e_1$ se obtiene a partir de $e_0$ por 0
o más renombres de ocurrencias de subfrases. También se dice que $e_0$
$\alpha$-convierte a $e_1$.

\smallskip

\textbf{Notación para expresiones $\alpha$- convertibles}: $e_0\equiv e_1$

\smallskip

\textbf{Redex:} Es una expresión de la forma $(\lambda v.e) e'$

\smallskip

\textbf{Contracción $\beta$}: Reemplaza en $e_0$ una
ocurrencia de un redex ($\lambda v.e) e'$ por su contracción
$(e/v\mapsto e')$, y luego efectúa cero o más renombres de cualquier
subexpresión.

\clearpage

\textbf{Notación}: Si $e_1$ es el resultado de una contracción $\beta$ de $e_0$, entonces escribimos
\[e_0\rightarrow e_1\]

\smallskip

\textbf{Forma normal:} expresión sin redices.
Las formas normales representan configuraciones terminales.
Por eso la semántica operacional del cálculo lambda consiste
en efectuar contracciones $\beta$ hasta obtener formas normales.

\bigskip
\bigskip

$\rightarrow^*$ denota la clausura transitiva y refexiva de $\rightarrow$

(o sea, aplicar $\rightarrow$ cero o más veces)

\smallskip

\textbf{Formalmente:}

$e\rightarrow^* e'$ si y sólo si existen $e_0,...,e_{n}$ (con $n\geq 0$) tales que:

\quad$e=e_0\rightarrow e_1\rightarrow ... \rightarrow e_{n}= e'$

Notar que si $n = 0$ entonces $e=e'$

\clearpage
\smallskip

\textbf{Teorema de Church-Rosser} Si  $e\rightarrow^* e_0$  y $e\rightarrow^* e_1$,
entonces existe $e'$ tal que $e_0\rightarrow^* e'$ y $e_1\rightarrow^*  e'$.

\begin{tikzpicture}
  \node (e) at (2,2) {$e$} ;
  \node (e0) at (0,0) {$e_0$} ;
  \node (e1) at (4,0) {$e_1$} ;
  \node (e2) at (2,-2) {$e'$} ;
  \draw [->,-latex] (e) -- node [at end,above=1mm] {$^\ast$} (e0);
  \draw [->,-latex] (e) -- node [at end,above=1mm] {$^\ast$} (e1) ;
  \draw [style=dashed,->,-latex] (e0) -> node [at end,above=1mm] {$^\ast$} (e2) ;
  \draw [style=dashed,->,-latex] (e1) -> node [at end,above=1mm] {$^\ast$} (e2) ;
\end{tikzpicture}

\textbf{Corolario 1.} Salvo renombre, toda expresión tiene a lo sumo
una forma normal.

\bigskip
\textbf{Regla $\eta$:} Un $\eta$-redex es una expresión de la forma $\lambda v.e v$, donde $v\notin FV\ e$

\begin{prooftree}
  \AxiomC{ } \RightLabel{si $v\notin FV\ e$\qquad ($\eta$)}
  \UnaryInfC{$\lambda v.e\, v \to e$}
\end{prooftree}

\clearpage

La idea de ejecución (llamada evaluación) que se implementa habitualmente
tiene las siguientes diferencias con la relación
\begin{itemize}
\item sólo se evalúan expresiones cerradas (es decir, sin variables libres)
\item es determin\'istica,
\item no busca formas normales sino formas canónicas.
\end{itemize}

\bigskip

\textbf{Evaluación (en orden) normal}: lenguajes funcionales lazy (Haskell)

\medskip

\textbf{Evaluación eager o estricta}: lenguajes estrictos (ML).

\bigskip

La noción de forma canónica depende de la definición de evaluación. Se define
una noción de forma canónica para la evaluación normal, y otra para la
evaluación eager. En el caso del cálculo lambda coinciden: \textbf{son las abstracciones}

\medskip

\textbf{Propiedad}: Una aplicación cerrada no puede ser forma normal. \pause

\bigskip

\textbf{Corolario}:  Una expresión cerrada que es forma normal es
también forma canónica.

\clearpage

\textbf{Semántica natural o big-step:}   En este tipo de semántica, uno
no describe un paso de ejecución,
sino directamente una relación entre los términos y sus valores (que
también son términos, son formas canónicas).Llamaremos $\Rightarrow$ a
esta relación.

\medskip
% ========== NORMAL ============ %
\textbf{Reglas para} $\Rightarrow_N$

Regla para las formas canónicas

$\begin{array}{c}
\overline{\lambda v.e\ \Rightarrow_N\ \lambda v.e}
\end{array}
$

\smallskip

Regla para la aplicación

\smallskip

$\begin{array}{c}
\underline{e\ \Rightarrow_N\ \lambda v.e_0\qquad (e_0/v\mapsto e')\ \Rightarrow_N\ z}\\
ee'\ \Rightarrow_N\ z
\end{array}
$
% ========== EAGER ============ %

\smallskip

\textbf{Reglas para} $\Rightarrow_E$

Regla para las formas canónicas

$\begin{array}{c}
\overline{\lambda v.e\ \Rightarrow_E\ \lambda v.e}
\end{array}
$

\smallskip

Regla para la aplicación

\smallskip

$\begin{array}{c}
\underline{e\ \Rightarrow_E\ \lambda v.e_0\qquad e'\ \Rightarrow_E\ z'\qquad(e_0/v\mapsto z')\ \Rightarrow_E\ z}\\
ee'\ \Rightarrow_E\ z
\end{array}
$

% =========================== %
\clearpage

% //// CALCULO LAMBDA DENOTACIONAL ///////////// %

Asumimos la existencia de un dominio $D_{\infty}$, junto con un
isomorfismo:

$\phi\in D_{\infty}\rightarrow [D_{\infty}\rightarrow D_{\infty}]$
\quad $\psi\in [D_{\infty}\rightarrow D_{\infty}]\rightarrow D_{\infty}$

\smallskip
$\phi \circ \psi = Id_{[D_{\infty}\rightarrow D_{\infty}]}$
y $\psi\circ\phi = Id_{D_{\infty}}$

\medskip

\textbf{Ambientes (Entornos):} $\eta \in Env = \langle var\rangle\rightarrow D_\infty $

\smallskip

\textbf{Función semántica:} $\se{\_}\in\ \ \langle exp\rangle \rightarrow Env\rightarrow D_{\infty}$

\smallskip


\textbf{Ecuaciones semánticas:}
\begin{align*}
    \se{v}\eta &= \eta v\\[1ex]
    \se{e_0 e_1}\eta &= \phi(\se{e_0}\eta)\ \se{e_1}\eta\\[1ex]
    \se{\lambda v.e}\eta &= \psi(\lambda d\in D_\infty. \se{e}[\eta|v:d])
\end{align*}

\smallskip

Se puede probar que $\se{\Delta\,\Delta}\eta = \bot$.

\textbf{Teorema de Coincidencia:}
Si $\eta w = \eta' w$ para todo $w\in FV\ e$, entonces
$\se{e}\eta = \se{e}\eta'.$

\smallskip

\textbf{Teorema de Renombre:} Si $v_{new}\notin FV\ e-\{v\}$,
entonces $\se{\lambda v_{new}.(e/v\mapsto v_{new})} = \se{\lambda v.e}.$ \pause

\smallskip

\textbf{Sustituciones:} $\Delta \ = \ \ \langle var\rangle \rightarrow  \langle exp\rangle$

\smallskip

\textbf{Teorema de Sustitución}: Si $\se{\delta w}\eta = \eta' w$ para todo $w \in FV\ e$, entonces $\se{e/\delta}\eta = \se{e}\eta'$.

\smallskip

\textbf{Propiedad 3.} \textit{(correctitud de la regla $\beta$):} $\se{(\lambda v.e)e'}\eta = \se{e/v\mapsto e'}\eta$

\smallskip

\textbf{Propiedad 4.} \textit{(correctitud de la regla $\eta$):} $\se{\lambda v.e\,v}\eta = \se{e}\eta, \text{ si } v\notin FV e$

\smallskip

\textbf{Corolario}: Si $e \rightarrow^\ast e'$, entonces $\se{e} = \se{e'}$.

\bigskip

\textbf{Semántica Denotacional Normal}

\smallskip
$D \ =\ V_\perp$, donde $V\ \ \approx\ \  [D\rightarrow D]$
\begin{align*}
\phi\in & V\rightarrow [D\rightarrow D] & \phi \circ \psi &= Id_{D\rightarrow D}\\
\psi\in &[D\rightarrow D] \rightarrow V & \psi\circ\phi &= Id_{V} \\
\end{align*}
\textbf{Notación: } $ \iota_\perp\in V\rightarrow D $

Dominio Semántico: $D \ = V_\perp \qquad  V\ \approx\ [D\rightarrow D]$

\smallskip

\textbf{Ambientes:} $Env = \langle var\rangle\rightarrow D$

\smallskip

\textbf{Función semántica:} $\se{\_}\in\ \ \langle exp\rangle \rightarrow Env\rightarrow D$

\textbf{Ecuaciones semánticas:}
\begin{align*}
  \se{v}\eta &= \eta v\\
  \pause
  \se{e_0 e_1}\eta &= \phi_{\perp\!\!\!\perp}(\se{e_0}\eta) (\se{e_1}\eta)\\
  \pause
  \se{\lambda v.e}\eta &=\  \iota_\perp\circ\psi\ (\lambda d\in D. \se{e}[\eta|v:d])
\end{align*}

\smallskip

Vale la regla $\beta$, que utiliza la igualdad:
\[\phi_{\perp\!\!\!\perp}\circ (\iota_\perp \circ \psi) = Id_{D\rightarrow D}\]


% ///////////////////// EAGER ///////////////// %
\textbf{Semántica Denotacional Eager}

Dominio Semántico: $D \ = V_\perp \qquad  V\ \approx\ [V\rightarrow D]$

\begin{align*}
\phi\in & V\rightarrow [V\rightarrow D] & \phi \circ \  &= Id_{V\rightarrow D}\\
\psi\in &[V\rightarrow D] \rightarrow V & \psi\circ\phi &= Id_{V}\\
\end{align*}

\textbf{Notación: } $\iota_\perp\in & V\rightarrow D $
\textbf{Ambientes:} $Env = \langle var\rangle\rightarrow V$

\textbf{Función semántica:} $\se{\_}\in\ \ \langle exp\rangle \rightarrow Env\rightarrow D $

\textbf{Ecuaciones semánticas:}
\begin{align*}
    \se{v}\eta &= \iota_\perp (\eta v)\\
    \pause
    \se{e_0 e_1}\eta &= (\phi_{\perp\!\!\!\perp}(\se{e_0}\eta))_{\perp\!\!\!\perp} (\se{e_1}\eta)\\
    \pause
    \se{\lambda v.e}\eta &=\  \iota_\perp\circ\psi\ (\lambda z\in V. \se{e}[\eta|v:z])
\end{align*}

Ya no vale la regla $\beta$, (para contra-ejemplo alcanza un $\bot$):

\[\se{(\lambda x\,y.y)\,(\Delta\,\Delta)}\eta = (\lambda z\in V. \se{e}[\eta|v:z])_{\perp\!\!\!\perp}\, \bot = \bot\]

pero
\[\se{(\lambda y.y)/x \mapsto(\Delta\,\Delta)}\eta = \se{\lambda y.y}\eta \neq \bot\]

\medskip

\pause Puesto que queremos modelar la evaluación eager, deberíamos
esperar que $e \Rightarrow_E z$ implique $\se{e}\eta = \se{z}\eta$.
\bigskip

Lo podemos probar por inducción en la derivación $e \Rightarrow_E z$.
\end{section}
